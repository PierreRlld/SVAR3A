% ============================================
% ============================================
\subsection{Method of \textit{Shapiro (2022)}}

The methodology proposed by Adam Hale Shapiro in his 2022 paper "Decomposing Supply and Demand Driven Inflation" takes its foundations in the work of Jump and Kohler (2022).
The core assumption of the framework is that for each sector/section $(i)$ of the inflation decomposition, an upward sloping in price supply curve and a downward demand curve can be assumed.

\begin{center}
    \begin{tikzpicture}
    \begin{axis}[
    xmin = 0, xmax = 10,
    ymin = 0, ymax = 10,
    axis lines* = left,
    xtick = \empty, ytick = \empty,
    clip = false,
    legend pos=outer north east
    ]
    %Labels
    \node [right] at (current axis.right of origin) {$P_{i}$};
    \node [above] at (current axis.above origin) {$Q_{i}$};
    % Lines
    \addplot[domain = 0:8, restrict y to domain = 0:10, samples = 400, color = red]{x+1};
    \addlegendentry{Supply}
    \addplot[domain = 0:8, restrict y to domain = 0:10, samples = 400, color = teal]{-0.7*x+7};
    \addlegendentry{Demand}
    \end{axis}
    \end{tikzpicture}
\end{center}

\noindent In theory, assuming $\sigma_{i}>0$ and $\delta_{i}>0$, we have:
\begin{align*}
    \textrm{Supply :} & \quad Q_{i} = \sigma_{i}.P_{i} + \alpha_{i}\\
    \textrm{Demand :} & \quad P_{i} = -\delta_{i}.Q_{i} + \beta_{i}
\end{align*}

\noindent Shocks are then defined as vertical movements of the curves :
\begin{align*}
    \textrm{Supply shock : } \varepsilon_{t}^{s} &= \Delta \alpha_{i} \; = (Q_{i,t} - \sigma_{i}P_{i,t}) - (Q_{i,t-1} - \sigma_{i}P_{i,t-1})\\
    \rightarrow \varepsilon_{t}^{s} &= \Delta Q_{i,t} - \sigma_{i}\Delta P_{i,t}\\
    \textrm{Demand shock : } \varepsilon_{t}^{d} &= \Delta \beta_{i} \; = (P_{i,t} + \delta_{i}Q_{i,t}) - (P_{i,t-1} + \delta_{i}Q_{i,t-1})\\
    \rightarrow \varepsilon_{t}^{d} &= \Delta P_{i,t} + \delta_{i}\Delta Q_{i,t}
\end{align*}

\noindent We can easily show that the two previous equations lead to :
\begin{align*}
    \Delta Q_{i,t} &= \frac{1}{1+\delta_{i}}(\varepsilon_{t}^{s} + \sigma_{i}.\varepsilon_{t}^{d}) \\
    \Delta P_{i,t} &= \frac{1}{\sigma_{i}(1+\delta_{i})}(\sigma_{i}.\varepsilon_{t}^{d} - \delta_{i}.\varepsilon_{t}^{s})
\end{align*}

\noindent As we assume $\sigma_{i}>0$ and $\delta_{i}>0$, we can derive expected comovements between the two variables following supply and demand shocks.
\begin{align*}
    \textrm{Supply shock } \; \Delta^{+}\varepsilon^{s} &: \; \Delta^{+}Q \quad \textrm{\&} \quad \Delta^{-}P &> \textrm{Negative comovements}\\
    \textrm{Demand shock } \; \Delta^{+}\varepsilon^{d} &: \; \Delta^{+}Q \quad \textrm{\&} \quad \Delta^{+}P &> \textrm{Positive comovements}
\end{align*}


\noindent Consider the following structural VAR of (arbitrary) order $p$ (dropping $i$ indices) and let $z_{t} = \begin{bmatrix} \Delta Q_{t} \\ \Delta P_{t} \end{bmatrix}$ and $\varepsilon_{t} = \begin{bmatrix} \varepsilon_{t}^{s} \\ \varepsilon_{t}^{d} \end{bmatrix}$:
\vspace*{.5cm}

\[
    A.\begin{bmatrix} \Delta Q_{t} \\ \Delta P_{t} \end{bmatrix} \;
    = \mu + \sum_{i=1}^{p}A_{i}.z_{t} + \begin{bmatrix} \varepsilon_{t}^{s} \\ \varepsilon_{t}^{d} \end{bmatrix}\;
\]

\vspace*{.4cm}
\noindent With $\nu_{t}$ the residuals of the estimated reduced-form VAR($p$) we should have $\nu_{t} = A^{-1}.\varepsilon_{t}$\\
Let $A$ satisfy $
    A \equiv\begin{pmatrix} 1 & - \alpha \\ \beta & 1 \end{pmatrix}
    $ 
with $\alpha, \beta > 0$, it follows that
$
A^{-1} = \frac{1}{1+\alpha \beta}\begin{pmatrix} 1 & \alpha \\ -\beta & 1 \end{pmatrix}
$\\
Omitting the $t$ indices we have : 
\begin{align*}
    \begin{bmatrix} \nu^{s} \\ \nu^{d} \end{bmatrix} &= A^{-1}.\begin{bmatrix} \varepsilon^{s} \\ \varepsilon^{d} \end{bmatrix} \\
    \begin{bmatrix} \nu^{s} \\ \nu^{d} \end{bmatrix} &= \frac{1}{1+\alpha \beta}\begin{pmatrix} 1 & \alpha \\ -\beta & 1 \end{pmatrix} \begin{bmatrix} \varepsilon^{s} \\ \varepsilon^{d} \end{bmatrix}
\end{align*}
As $\frac{1}{1+\alpha \beta} > 0$, we finally have:

\begin{align*}
    \nu^{s} &\propto \varepsilon^{s} + \alpha.\varepsilon^{d} \\
    \nu^{d} &\propto -\beta.\varepsilon^{s} + \varepsilon^{d}
\end{align*}






% ============================================
% ============================================
\subsection{Method of \textit{Sheremirov (2022)}}